\chapter{Introduction}

The study assignment is a transcript analysis of my presentation about \enquote{Ensemble Learning}. The purpose is  to give a brief overview of the technical part of the presentation and afterwards to focus on the beginning, the end or another section of the main part of the presentation. The approach is to reflect what was produced in the real-time presentation. In addition the study assignment includes a one page analysis of the transcript in which a detailed breakdown of the presentation language and techniques/elements, which have been used and why they have been used, is given. Last but not least, the study assignment finishes with a conlusion section which can be seen as a phenomenological reflection on the subjective learning experience across all phases and challanges of the presentation process. Furthermore a short introduction to what \textit{Ensemble Learning} is. It is a system of multiple classifiers or also known as \enquote{committee-based learning} \autocite[]{Zhou.2021}, which trains and combines certain learners and therefore learning algorithms to achieve a higher generalization ability. The three most common and used methods are \enquote{Stacking}, \enquote{Bagging} and \enquote{Boosting} \autocite[]{Zhou.2021} \autocite[]{Zhou.2012}.