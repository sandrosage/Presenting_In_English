\chapter{Main Transcript}

\section{PART I}

[\dots] I'm working on embedded machine learning in a lab on the side and this is exactly the reason why I can tell you something about machine learning and therefore also ensemble learning. First, a few organizational things. The presentation will last approximately 14 minutes and if there are any questions, please take notes and ask your questions at the end of the presentation. To get directly into the topic, there are two cartoons on the next slide.I would like you to look at both of them and then try to explain what ensemble learning can be? [\dots] But before I explain what ensemble learning is, let's have a quick look at the roadmap of the presentation. First, [\dots] summary of what has been said so far, so that the most important points remain in memory [\dots] Okay fine, but what is bootstrap sampling? Its a common method for dividing data into sometimes overlapping subsets.[\dots] As you can see in the picture on the right, imagine you have one big dataset and three individual learners, then you have to divide the whole dataset into three bootstrapped subsets and train the learners on each of them. The reason for this is that bagging uses homogeneous ensembles, and you cannot train all learners based on the same data, otherwise every algorithm would learn the same things and act in the same way. One applied example is Random Forest, so that's not a forest somewhere in the nowhere but a bagging ensemble of decision trees. [\dots] If you compare this method with the others and in addition look at the diagram on the right, you will immediately notice that this is a sequential approach. [\dots] Boostings goal is to convert weak learners into strong learners, this is also where the term comes from, because weak learners are boosted.  It is an iterative process with training learners and adjustment of training samples. But what is meant by adjustment? So therefore, lets take another look at the diagram. Imagine you have a sample distribution belonging to two classes: blue and orange. Then you want to train your individual learner on this sample distribution. After the training you will get some misclassified samples, but how can you improve your learning algorithm to perform better? This is where sample adjustment comes in. You assign a weight to every sample and each misclassified sample will get a higher weight. In the diagram, this is illustrated by expanding the dot size of one sample. [\dots]

\section{PART II}
First of all the presentation starts with introducing myself and giving an overview of the \enquote{Presentation Journey}\autocite[]{williams2008presentations}. The aim is to show the audience what I am currently working on, what my specialisation is and why I am the right one to tell the audience something about \textit{Ensemble Learning}. (\enquote{I' working on embedded machine learning \dots this is exactly the reson why I can tell you something about \dots ensemble learning}). This should directly get the attention of the audience and show the audience that I am a professional presenter. Further on, I want to attach the attention of the audience once again with a direct question to involve the audience (\enquote{\dots try to explain what ensemble learning can be?}). The background idea is the \enquote{BANG}-effect \autocite[]{bomberb} and  to simply get the audience think about the topic with the aid of cartoons (\enquote{\dots there are twocartoons \dots}). Additionally, I'm using the verb \enquote{look} to activate the \textit{visual} sense of the \enquote{VAKOG}-system (\enquote{I would like you to look at \dots}; \enquote{Let's have a quick look at \dots}). A general structure for the presentation is also provided by using the \textit{Presentation Journey}, as I mentioned earlier, and some language phrases in \citetitle{emmerson2007business}. Considering the presentation transcript, example phrases were applied for the roadmap and logistics aspects of \textit{Presentation Journey} (\enquote{Let's have a quick look at the roadmap \dots}; \enquote{The presentation will last \dots 14 minutes and if you have any questions \dots}). Another very important method called \enquote{Signposting} \autocite{williams2008presentations} was utilized (\enquote{First, \dots}). Moreover I then explained the different methods of \textit{Ensemble Learning}, which are rather theoretical, so with the help of the \enquote{VAKOG}-system \autocite{JumbuhPrabowo.2015} once again, I tried to explain and illustrate the content in a more conceivable and comprehensible way (\enquote{As you can see \dots imagine you have a big dataset \dots}). In addition the usage of rhetorical questions is a key strategy to explain complex technical expressions. The reason for this is to encourage the audience to ask themselves the question in their minds and furthermore to become even more curious and attentive (\enquote{Okay fine, but what is bootstrap sampling?}). Another thought is that presentations are always built up with bullet points and a lot of information, so a rhetorical question can revive the information once again. It's always hard to explain difficult theoretical topics without any visualizations thus I used different diagrams and pictures to illustrate the content (\enquote{As you can see in the picture on the right \dots}). For the explanation and overview of the different methods of \textit{Ensemble learning}, I'm giving applied examples in the real world, so that interested people in the audience can easily remember the examples and then make further research. Since the information is very tough and rather theoretical, the usage of humorous phrases is a key element to keep the adience intent and prevent them from being overwhelmed or lost in the information (\enquote{\dots Random Forest,so that's not a forest somewhere in the nowhere \dots}). With another rhetorical question \enquote{What is meant by adjustment?}, I can moreover explain what is meant by the expression and how it works. Additionally, I tell the audience that we therefore should have another look at the diagram presented in this slide to also have an illustrative way of explanation (\enquote{So therefore, lets take another look at the diagram.}). Apart from this the \textit{Vakog}-system more specifically the \textit{visual} sense is used to to achieve a better imagination in the minds of the listeners and thus get the gist of the content (\enquote{Imagine you have a sample distribution belonging to two classes \dots}). With regard to this I'm always using \enquote{you} to directly address the audience and achieve a higher imagination once again.