\chapter{Conclusion}

In conclusion, the presentation and the course in general have given me a different perspective on the way presentations can be constructed and designed. I have already given so many presentations without directly thinking about the effects and impacts of the individual techniques. My main take away is that I can not only convey the information, but I now know to keep the audience curious and awake, how I have to build my structure through the presentation in a meaningful way (e.g. \enquote{BOMBER B} \autocite[]{bomberb}) and how I can easily reflect the difficult-to-unterstand technical aspects through targeted word choice (e.g. \textit{VAKOG}). Another insight which I gained from my presentation is the use of rhetorical questions. One example for this was the question \enquote{Okay fine, but what actually is bootstrap sampling?}, and with this approach, I could simply illustrate to the audience what exactly is meant by the technical expression und revive the conent once again. My biggest challenge was the tough information, which was very theoretical and difficult to unterstand, and the resulting question: How can I explain and present it in a simplified and clear way? Presentations that contain a lot of theory are usually boring, that's the reason why I used a \textit{BANG} to involve the audience. The other key for mastering this challenge is the usage of the \textit{VAKOG}-System to transform the information in a more illustrative and comprehensible way. Another important method is \textit{Signposting}. Before the course and the presentation, I always created a presentation with different parts and sections without being aware that there is such a technique like \textit{Signposting} and what it's used for. \textit{Signposting} creates \enquote{verbal paragraphs} or \enquote{verbal singals} to raise the attention curve at the beginning and end of each point of your presentation. Therefore, in combination with the \textit{Presentation Journey}, \textit{Signposting} is the best way to structure and shape a presentation. All in all, I am now not only able to give an informative talk, but I am also aware of what I am doing and the effect I am having on the audience.